
\chapter{Beispiele}
\section{Eine seltsame Funktion}\label{sec:math}

Gegeben ist die Funktion $f: \mathbb{R} \to \mathbb{R}$ definiert als
\begin{equation}\label{eq:fundef}
	f(x) = \left\{
		\begin{array}{ll}
			0 & \text{für } x \in \mathbb{R} \\
			\frac{1}{q} & \text{für } x \in \mathbb{Q} \text{ mit }
			x=\frac{p}{q},\, q \in \mathbb{N},\, p \in \mathbb{Z},\,
			\text{ggT}(p,q)=1.
		\end{array} \right.
\end{equation}

\begin{thm}\label{thm:seltsam}
Die Funktion $f$ definiert in \autoref{eq:fundef} ist stetig für
irrationale $x \in \mathbb{R}\setminus \mathbb{Q}$ und unstetig für 
rationale $x \in \mathbb{Q}$.
\end{thm}

\subsection{Beweis von \autoref{thm:seltsam}}

\begin{lem}\label{lem:lemma1}
Sei $x_i$ eine Folge rationaler Zahlen, die gegen eine
	irrationale Zahl $r$ konvergiert. Schreibt man 
	$x_i=\frac{p_i}{q_i}$ mit 
	$p_i \in \mathbb{Z}$ und $q_i \in \mathbb{N}$ gilt
	\[
		\lim_{n \to \infty} q_i = \infty
	\]
\end{lem}

\begin{proof}[Beweis von \autoref{lem:lemma1}]

	Wir können annehmen, dass $\text{ggT}(p_i, q_i)=1$, d.h.
	die Brüche sind vollständig gekürzt. Der Beweis 
	wird durch Widerspruch geführt: Nehmen wir an 
	\[
		q_i<N\, \forall i.
	\]

	Für alle $x_i \neq x_j$ gilt $|x_i - x_j| \geq \frac{1}{N^2}$ weil
	\[
		\left|\frac{a}{b} - \frac{c}{d} = \frac{ad-bc}{bd}\right| \geq
		\frac{1}{bd} \geq \frac{1}{N^2}
	\]
	Da die Folge $(x_i)$ gegen $r$ konvergiert, gibt es für jedes $\epsilon>0$
	ein Index $I$, ab dem gilt:
	\[
		|x_i-r|<\epsilon \quad \forall i>I
	\]
	Wählt man $\epsilon<\frac{1}{3N^2}$ folgt aus $|x_i - x_j| \geq
	\frac{1}{N^2}$,
	dass 
	\[
		x_i = x_j \quad \forall i,j > I
	\]
	Damit ist aber auch $r=x_i$ für $i>I$, was ein Widerspruch zur Annahme ist,
	dass $r$ irrational ist.
	
\end{proof}


\subsubsection{Stetigkeit für irrationale Argumente}
\begin{proof}
Wie in \autoref{lem:lemma1} gezeigt, sind,
um eine irrationale Zahl $r$ mit Brüchen
anzunähern, immer grössere Nenner $q_i$ nötig:
\[
	\lim_{n\to \infty} q_i = \infty
\]
und damit
\[
	\lim_{n \to \infty} \frac{1}{q_i} = 0,
\] 
was die Stetigkeit in irrationalen Argumenten $r$ beweist.
\end{proof}

\subsubsection{Unstetigkeit für irrationale Argumente}
\begin{proof}
Für jede rationale Zahl $q$ ist $f(q)>0$. 
Für jedes $\epsilon>0$ existiert 
eine irrationale Zahl $r$ mit $|r-q|<\epsilon$.
Weil $f(r)=0$ ist $f$ in $q$ unstetig.
\end{proof}


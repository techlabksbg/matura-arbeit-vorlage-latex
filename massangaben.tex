\chapter{Massangaben}\label{sec:massangaben}

Sobald Masseinheiten angegeben werden, lohnt sich das package
{\tt siunitx}\cite{siunitx}.


\section{Beispiele}
Mehr Beispiele finden Sie in der offiziellen Dokumentation des package.
\begin{itemize}
	\item Beschleunigung $g = \qty{9.81}{\metre\per\square\second}$. 
	\verb+$g = \qty{9.81}{\metre\per\square\second}$+
%
	\item Strom $I = \qty{25}{\micro\ampere}$.
		\verb+$I = \qty{25}{\micro\ampere}$+
	\item Widerstand $R = \qty{2}{\mega \ohm}$.
		\verb+$R = \qty{2}{\mega \ohm}$+
\end{itemize}

\section{Alte Version (v2) von siunitx}
Sollte Ihr System noch eine alte Version von siunitx haben, verwenden Sie
anstatt \verb+\qty+ den Befehl \verb+\SI+, also z.B.

\verb+\SI{25}{\micro\ampere}  +
anstatt
\verb+   \qty{25}{\micro\ampere}+

Siehe auch \cite{qty-vs-si}.

\chapter{Mathematik in \LaTeX}\label{sec:mathezeugs}
Es empfiehlt sich die <<amsmath>> package einzubinden, mit
\begin{verbatim}
\usepackage{amsmath}
\end{verbatim}

\section{Gleichungen}
Je nach Layout gibt es verschiedene Umgebungen. Gute Beispiele finden sich auch
in \LaTeX{}-Dokumentation von Overleaf\cite{overleaf-equations}.
\subsection{Einzeiler}
Normalerweise kriegen alle Gleichungen eine Nummer wie die
\autoref{eq:erstes-beispiel} und werden im Text referenziert.
\begin{equation} \label{eq:erstes-beispiel}
	B > \frac{1}{n} \sum_{i=1}^n x_i
\end{equation}
%
Die \autoref{eq:erstes-beispiel} wurde durch den Code in
\autoref{lst:erstes-beispiel} erzeugt:
%\begin{latexcode}[Code zu \autoref{eq:erstes-beispiel}]%[erstes-beispiel]
\begin{latexcode}[{Code zur \autoref{eq:erstes-beispiel}}]{lst:erstes-beispiel}
\begin{equation} \label{eq:erstes-beispiel}
    B > \frac{1}{n} \sum_{i=1}^n x_i
\end{equation}
\end{latexcode}


\noindent Soll eine Gleichung oder Formel ausnahmsweise keine Nummer erhalten wie in
\[
	E = mc^2,
\]
kann der Code in \autoref{lst:keine-nummer} verwendet werden.
\begin{latexcode}[{Nicht nummerierte Gleichung}]{lst:keine-nummer}
\[
    E = mc^2
\]
\end{latexcode}

\subsection{Mehrzeiler}
Um Umformungen von Gleichungen darzustellen, 
eigenet sich die {\tt aligned} Umgebung, wie in 
\autoref{eq:gleichungumformen} zu sehen:
\begin{equation} \label{eq:gleichungumformen}
	\begin{aligned}
		-\frac{6}{7}x = & -\frac{10}{3}+\frac{4}{7}x & & | -\frac{4}{7}x\\
		-\frac{10}{7}x  = & -\frac{10}{3} & & | : -\frac{10}{7}\\
		x  = & \frac{7}{3} \\
	\end{aligned}
\end{equation}
Der Code dazu gibt es in \autoref{lst:gleichungumformen}.
Ganz ähnlich wie {\tt aligned} funktioniert die {\tt align}-Umgebung 
(alleine, nicht innerhalb einer {\tt equation} Umgebung), 
die dann aber jede
Zeile durchnummeriert, was durchaus auch nützlich sein kann.
\begin{latexcode}[Code zur \autoref{eq:gleichungumformen}]{lst:gleichungumformen}
\begin{equation} \label{eq:gleichungumformen}
    \begin{aligned}
        -\frac{6}{7}x = & -\frac{10}{3}+\frac{4}{7}x & & | -\frac{4}{7}x\\
        -\frac{10}{7}x  = & -\frac{10}{3} & & | : -\frac{10}{7}\\
        x  = & \frac{7}{3} \\
    \end{aligned}
\end{equation}
\end{latexcode}


%
Sollen einfach längere Umformungen dargestellt werden
eignet sich die {\tt multline} Umgebung:
\begin{multline} \label{eq:langeumformung}
	f'(x) = \lim_{h \to 0} \frac{f(x+h)-f(x)}{h} = 
	\lim_{h \to 0} \frac{a^{x+h}-a^x}{h} =  \\
	\lim_{h \to 0} \frac{a^x \cdot a^h-a^x}{h} = 
	\lim_{h \to 0} \frac{a^x \cdot \left(a^h-1\right)}{h} =  \\
	a^x \cdot \lim_{h \to 0} \frac{a^{0+h}-a^0}{h} = a^x \cdot f'(0)	
\end{multline}

Der Code zu \autoref{eq:langeumformung} ist in \autoref{lst:langeumformung}
dargestellt.
\begin{latexcode}[Code zur \autoref{eq:langeumformung}]{lst:langeumformung}
\begin{multline} \label{eq:langeumformung}
    f'(x) = \lim_{h \to 0} \frac{f(x+h)-f(x)}{h} = 
    \lim_{h \to 0} \frac{a^{x+h}-a^x}{h} =  \\
    \lim_{h \to 0} \frac{a^x \cdot a^h-a^x}{h} = 
    \lim_{h \to 0} \frac{a^x \cdot \left(a^h-1\right)}{h} =  \\
    a^x \cdot \lim_{h \to 0} \frac{a^{0+h}-a^0}{h} = a^x \cdot f'(0)
\end{multline}
\end{latexcode}

\section{Vektoren}

\begin{equation}\label{eq:erste}
\vec v = \begin{pmatrix}
a_1\\
a_2\\
a_3
\end{pmatrix}
\end{equation}

Verwendet man \autoref{eq:erste} dann erhält man:

\begin{equation}\label{eq:zweite}
\vec u = \begin{pmatrix}
a_1+b_1\\
a_2-\pi\\
a_3+\cos\left(\alpha^2\right)
\end{pmatrix}
\end{equation}

Der Code dazu ist in \autoref{lst:latex-code} ersichtlich.

\begin{latexcode}[Code zu \autoref{eq:erste} und \autoref{eq:zweite}]{lst:latex-code}
\begin{equation}\label{eq:erste}
\vec v = \begin{pmatrix}
a_1\\
a_2\\
a_3
\end{pmatrix}
\end{equation}

Verwendet man \autoref{eq:erste} dann erhält man:

\begin{equation}\label{eq:zweite}
\vec u = \begin{pmatrix}
a_1+b_1\\
a_2-\pi\\
a_3+\cos\left(\alpha^2\right)
\end{pmatrix}
\end{equation}

Den Code dazu ist in \autoref{fig:latex-code} ersichtlich.
\end{latexcode}
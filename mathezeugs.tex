\chapter{Mathematik in \LaTeX}\label{sec:mathezeugs}

\section{Vektoren}

Es empfiehlt sich die <<amsmath>> package einzubinden, mit
\begin{verbatim}
\usepackage{amsmath}
\end{verbatim}

\begin{equation}\label{eq:erste}
\vec v = \begin{pmatrix}
a_1\\
a_2\\
a_3
\end{pmatrix}
\end{equation}

Verwendet man \autoref{eq:erste} dann erhält man:

\begin{equation}\label{eq:zweite}
\vec u = \begin{pmatrix}
a_1+b_1\\
a_2-\pi\\
a_3+\cos\left(\alpha^2\right)
\end{pmatrix}
\end{equation}

Den Code dazu ist in \autoref{fig:latex-code} ersichtlich.

\subsection{Code}
\begin{figure}[ht]
\centering
\begin{minipage}{0.8\textwidth}
\begin{verbatim}
\begin{equation}\label{eq:erste}
\vec v = \begin{pmatrix}
a_1\\
a_2\\
a_3
\end{pmatrix}
\end{equation}

Verwendet man \autoref{eq:erste} dann erhält man:

\begin{equation}\label{eq:zweite}
\vec u = \begin{pmatrix}
a_1+b_1\\
a_2-\pi\\
a_3+\cos\left(\alpha^2\right)
\end{pmatrix}
\end{equation}

Den Code dazu ist in \autoref{fig:latex-code} ersichtlich.

\end{verbatim}
\end{minipage}
\caption{Code zu \autoref{eq:erste} und \autoref{eq:zweite}}
\label{fig:latex-code}
\end{figure}

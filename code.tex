\chapter{Einbinden von Code}\label{sec:code}
Vollständige Code-Listings gehören nicht in den Bericht, sondern 
die dem Bericht beigelegte SD-Karte.\footnote{Wobei das nicht mehr erlaubt ist,
wie die Codes der Arbeit archiviert werden sollen, erschliesst sich
mir nicht.}

Schlüsselstellen in einem Programm machen aber durchaus Sinn,
aufgeführt zu werden. Ein Beispiel für Python-Code 
finden Sie im \autoref{lst:python-code}.

Für \LaTeX-Code finden Sie ein Beispiel im \autoref{lst:figure-example}.

Im \autoref{lst:javaquine} auf der nächste Seite ist das
Listing eines Quines abgebildet (ein Programm, das seinen eigenen Source-Code als
Ausgabe produziert).

\section{Python Code}
\lstinputlisting[
label=lst:python-code,
language=python,
caption={Python-Code zur Erzeugung des Pascaldreiekcs},
]{code/pascal.py}

\newpage
\section{Java Code}
\lstinputlisting[
label=lst:javaquine,
caption={[JavaQuine]Ein Java Quine\cite{javaquine}},
% linerange={88-93},firstnumber=88
]{\codefile{Quine}}

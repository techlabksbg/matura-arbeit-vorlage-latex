\chapter{Darstellung}
Die Titel vom Kapitel und Unterkapitel werden
in der Kopfzeile angezeigt. Wenn die zu lange sind, können alternative, kürzere Titel für die Kopfzeile und das Inhaltsverzeichnis verwendet werden.
\section[Gekürzter Titel]{Verdammt langer Titel, der nicht in die Kopfzeile passt}

Anfrage an ChatGPT:
	\begin{quote}
When writing a report with documentclass$\{$report$\}$, how can I prevent long section titles from overflowing the page header?
	\end{quote}
\noindent
Und die Antwort:
	\begin{quote}
To prevent long section titles from overflowing the page header in a LaTeX
		document using the report document class, you can use the short optional
		argument for the $\backslash$section command to specify a shorter version of the section title to be used in the page header. Here's an example (\autoref{lst:shorttitleoption}):

\begin{latexcode}[Optionaler kurzer Titel]{lst:shorttitleoption} % Das ist in vorlage.txt in Zeile 133 definiert
\documentclass{report}
\begin{document}
\section[Short Title]{Long Section Title That May Overflow the Header}
Text of the section goes here.
\end{document}
\end{latexcode}
In this example, the short version of the section title is specified within square brackets []. This short title will be used in the page header instead of the full title.

\newpage
Alternatively, you can manually set the header using the $\backslash$markboth command. Here's an example (\autoref{lst:markboth}):
\begin{latexcode}[Beide Titel (links und rechts) überschreiben]{lst:markboth}
\section{Long Section Title That May Overflow the Header}
\markboth{Short Title}{Short Title}
\end{latexcode}

In this example, $\backslash$markboth sets both the left and right page headers to "Short Title", effectively overriding the automatic section title inclusion. You can adjust the short title to your liking.

	\end{quote}
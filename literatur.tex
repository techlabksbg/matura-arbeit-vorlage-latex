\chapter{Literatur und Bib\TeX} \label{ch:literatur}
\section{Literatur-Verweise}
Literaturangaben in \LaTeX{} \cite{lamport-latex} werden in einem
separaten <<.bib>>-File gespeichert und dann im Code mit <<{\tt
  $\backslash$cite\{{\em name}\}}>> eingebunden. 
Es werden am Schluss automatisch nur jene aufgeführt, die auch tatsächlich im
Text referenziert wurden. Überschüssige Literaturangaben in der .bib-Datei 
stören nicht und sollten auch nicht gelöscht werden, die könnten später noch
einmal nützlich sein.

Zusätzliche können beim Verweis noch die Seitenzahlen angegeben werden, wie 
im folgenden Beispiel:
\begin{quote}
Es folgt eine mathematisch strenge Herleitung, die im Jahre 1953 in einer russischen Mathematikzeitschrift
	veröffentlicht wurde~\cite[181-182]{formula-derivation}.
\end{quote}
Die Referenz wird mit 
\begin{verbatim}
\cite[181-182]{formula-derivation}
\end{verbatim}
eingebunden. Der Eintrag in der .bib-Datei (in diesem Fall die Datei
literatur.bib) sieht wie folgt aus:

\begin{verbatim}
@book{formula-derivation,
    author={A. M Yaglom and I. M. Yaglom},
    title= {An elementary derivation of the formulas of Wallis, 
	                      Leibniz and Euler for the number pi.},
    publisher = {Uspekhi Mat. Nauk},
    volume={Volume 8},
    year={1953}
}
\end{verbatim}

\subsection{Die .bib-Datei}
Es lohnt sich, zuerst online nach einem fertigen Eintrag zu
suchen. Das ist oft schneller und bequemer, als selbst einen zu
erstellen. Es gibt auch Programme, die
leere Einträge erstellen. Oder man kopiert einen ähnlichen
Eintrag und ändert diesen ab.
Mehr zum Format der Einträge online \cite{bibtex}.


\subsection{ISBN to Bib\TeX{}Converter}
Mit online <<ISBN to Bib\TeX{} Converter>>
\cite{isbn2bibtex} können auch ISBN-Nummern direkt in
Bib\TeX{}-Einträge übersetzt werden (der Amazon-Link darf dabei
hemmungslos gelöscht werden).

\subsection{Kompilierung}
Diese Vorlage verwendet <<biber>>\cite{bibtex-with-biber}. Das Programm muss
eventuell noch separat installiert werden.

\newpage
\section{Interne Verweise} \label{sec:verweise-intern}
Nach jedem Befehl, der eine Nummer erzeugt, 
kann ein <<label>> platziert werden, den 
dann mit <<autoref>> referenziert werden kann, 
siehe \autoref{fig:verweise-intern}.


\begin{figure}[ht]
\centering
\begin{minipage}{0.8\textwidth}
\begin{verbatim}
\section{Interne Verweise} \label{sec:verweise-intern}
Nach jedem Befehl, der eine Nummer erzeugt, 
kann ein <<label>> platziert werden, den 
dann mit <<autoref>> referenziert werden kann, 
siehe \autoref{fig:verweise-intern}.

\begin{figure}[ht]
\centering
\begin{minipage}{0.8\textwidth}

	...

\end{minipage}
	\caption{Code-Beispiel zu \autoref{sec:verweise-intern}}
	\label{fig:verweise-intern}
\end{figure}
\end{verbatim}
\end{minipage}
	\caption{Code-Beispiel zu \autoref{sec:verweise-intern}}
	\label{fig:verweise-intern}
\end{figure}




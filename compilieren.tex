\chapter{Kompilierung}\label{sec:compilieren}

\section{Technischer Ablauf}
\LaTeX-Dateien (Dateiendung \texttt{.tex})
sind einfache Text-Dateien (d.h. nur Buchstaben, keine
Formatierung). 
Sie können mit einem beliebigen Text-Editor bearbeitet werden.
Spezialiserte Editoren oder Modi sind aber empfehlenswert.\\
%
Aus den \LaTeX-Dateien wird dann eine PDF-Datei erzeugt.
Diesen Vorgang nennt man {\em kompilieren}.\\
%
Bei der Kompilierung gibt es mehrere Schritte, die bis auf einen
vom Programm \texttt{pdflatex} erledigt werden.
\begin{itemize}
	\item Setzen des Text und <<Sammeln>> der Verweise.
	\item Abermaliges Setzen und Einfügen der Verweise.
	\item Setzen des Literaturverzeichnisses 
		(hier mit \texttt{biber}, früher mit \texttt{bibtex}).
	\item Abermaliges Setzen und Einfügen der Verweise.
\end{itemize}
Je nach Editor müssen diese Schritte manuell angestossen werden
oder erfolgen automatisch (z.B. mit \TeX{}Studio) oder dem 
beiliegenden \texttt{Makefile}, siehe \autoref{sec:make}.

\section{Installation der Software}

\subsection{Windows}\label{sec:windows}
Für alle \LaTeX{}-Pakete und Programme zur Kompilierung wird
MiK\TeX{} empfohlen, hier zu finden:
\begin{center}
\url{https://miktex.org/download}
\end{center}
\noindent Bei der Installation empfehle ich alle default-Einstellung, bis
auf die Einstellung 
<<{\em Pakete ohne Nachfrage automatisch nachinstallieren}>>, die
ich empfehlen kann.

Nach der Installation soll gerne nach Updates gesucht werden. Diese werden nach
ca. 1 min auch zur Installation angeboten und sollten dann eingespielt werden.


\noindent Als Editor kann ich \TeX{}studio empfehlen, hier zu finden:

\begin{center}
\url{https://www.texstudio.org/}
\end{center}

\noindent Von \TeX{}live und \TeX{}works kann ich nicht viel Gutes berichten.


\subsection{Mac}
Da gibt es offenbar \href{https://tug.org/mactex/}{MacTeX
  (https://tug.org/mactex/)}. Wird wohl gut sein.

\subsection{Cloud-Lösung Overleaf}
Cloud-Lösung für \LaTeX. Sorgen Sie für ein regelmässiges lokales
Backup (wie bei allen Cloud-Diensten sind Ihre Daten auf fremden Datenträgern).

\subsection{Linux}\label{sec:linux}
Für Ubuntu sollten folgende Pakete installiert werden:\\[2mm]
%
\texttt{
sudo apt install texlive-latex-base texlive-pictures texlive-science\\ 
sudo apt install texlive-latex-extra texlive-lang-german biber
}\\[2mm]
Hinweis: \TeX{}studio gibt es auch für Linux.

\section{Kompilierung mit make}\label{sec:make}
Das Programm \texttt{make} erlaubt es, Abhängigkeiten zu definieren
wie aus welchen Dateien weitere erzeugt werden sollen. Das wird normalerweise
in der Softwareentwicklung verwendet, eignet sich aber auch für die 
Kompilierung dieses Dokuments.\\
%
Das Programm \texttt{make} kann durch Installation des 
folgendes Pakets installiert werden:

\texttt{
	sudo apt install build-essential
}

\noindent Für diese Vorlage wurde das Makefile in \autoref{lst:makefile} verwendet.

\lstinputlisting[
	language=make,
	label=lst:makefile,
	caption={[Makefile]Das Makefile, das diese Dokument kompiliert.},
% linerange={88-93},firstnumber=88
]{Makefile}


Die einzelnen Kommandos können natürlich auch manuell eingegeben
werden. Gerade für grössere Projekte eignen sich Makefiles aber, da
alle Abhängigkeiten spezifiziert werden können. Z.B.\ können auch
programmatisch erzeugte Grafiken automatisch neu erstellt werden, wenn
die zugrundeliegenden Daten sich ändern.


\chapter{Kompilierung}\label{sec:compilieren}

\section{Windows}\label{sec:windows}

Für Windows kann 
\href{https://tug.org/texlive/windows.html}{\TeX{}live} empfohlen werden,
Dowload auf \href{https://tug.org/texlive/windows.html}{\url{https://tug.org/texlive/windows.html}}

Normalerweise wird \LaTeX{} mit einem Texteditor geschrieben,
vorzugweise mit einem speziellen \LaTeX{}-Modus, oder gleich ein
spezialiserter Texteditor. \TeX{}live empfiehlt den \TeX{}works Editor,
zu finden auf 
\href{https://www.tug.org/texworks/}{\url{https://www.tug.org/texworks/}}.

Eine weitere Möglichkeit für eine LaTeX-Distribution
ist MiKTeX, die auch mit einem 
spezialisierten Editor geliefert wird. 

Es gibt auch ein <<Grafisches Interface>>,
\href{https://www.lyx.org/}{LyX (http://www.lyx.org)}. Was der taugt,
und ob das überhaupt empfehlenswert ist, kann ich nicht beurteilen.

\section{Mac}
Da gibt es offenbar \href{https://tug.org/mactex/}{MacTeX
  (https://tug.org/mactex/)}. Wird wohl gut sein.

\section{Cloud-Lösung Overleaf}
Cloud-Lösung für \LaTeX. Sorgen Sie für ein regelmässiges lokales
Backup (wie bei allen Cloud-Diensten sind Ihre Daten auf fremden Datenträgern).

\section{Linux}\label{sec:linux}
Für diese Vorlage wurde das Makefile in \autoref{lst:makefile} verwendet.

Sollte das Programm make nicht schon installiert sein, installieren Sie bitte das Ubuntu-Package 
<<build-essential>> (oder ähnliches).

\lstinputlisting[
	language=make,
	label=lst:makefile,
	caption={[Makefile]Das Makefile, das diese Dokument kompiliert.},
% linerange={88-93},firstnumber=88
]{Makefile}


Die einzelnen Kommandos können natürlich auch manuell eingegeben
werden. Gerade für grössere Projekte eignen sich Makefiles aber, da
alle Abhängigkeiten spezifiziert werden können. Z.B.\ können auch
programmatisch erzeugte Grafiken automatisch neu erstellt werden, wenn
die zugrundeliegenden Daten sich ändern.

